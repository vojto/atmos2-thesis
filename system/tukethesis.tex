\nonstopmode

\documentclass[]{tukethesis}
%% -----------------------------------------------------------------
%% UTF-8 ENCODING USED. USE PDFLATEX TO COMPILE YOUR DOCUMENTS.
%% -----------------------------------------------------------------
\usepackage{listings}
\usepackage{courier}
\lstset{basicstyle=\footnotesize\ttfamily,breaklines=true}
\lstset{framextopmargin=50pt,frame=bottomline,captionpos=b}
\usepackage[slovak,english]{babel}
\usepackage[utf8]{inputenc}
\usepackage[T1]{fontenc}
\usepackage{cmap}
%% ---- definicia slovenskych uvodzoviek
\chardef\clqq=18 \sfcode18=0
\chardef\crqq=16 \sfcode16=0
\def\uv#1{\clqq#1\crqq}
%% ------------------------------------
\renewcommand{\figurename}{Fig.}
\renewcommand{\tablename}{Tab.}
\renewcommand{\refname}{Bibliography}
\renewcommand{\listfigurename}{List of Figures}
\renewcommand{\listtablename}{List of Tables}
\renewcommand{\contentsname}{Contents}
%%
\usepackage{latexsym}
\usepackage{dcolumn} % alignment on a 'decimal point' in tabular mode
\usepackage{hhline}
\usepackage{amsmath}
\usepackage{nicefrac} % nice fractions
\usepackage{upgreek} % e.g. $\upmu\mathrm{m}$ type micrometer (mu)
\usepackage[final]{showkeys}%color%notref%notcite%final
\usepackage[noprefix]{nomencl}
\usepackage{verbatim}
\makeglossary % command to make *.glo file
% \usepackage{parskip}
\usepackage{indentfirst}
% \usepackage{listings}

\usepackage{caption}
\captionsetup[lstlisting]{font={bf,footnotesize}}
\captionsetup[figure]{font={bf,footnotesize}}
%%
%\usepackage[dvips]{graphicx}
%\DeclareGraphicsExtensions{.eps, .mps}
\usepackage[pdftex]{graphicx}
\DeclareGraphicsExtensions{.pdf,.png,.jpg,.mps}
% \graphicspath{{figures/}} % directory for figures
%%
%% numerical citations (vancouer style)
\usepackage[numbers]{natbib}
%%
%% author-year citations (harvard style) -- prefered !!!
%\usepackage{natbib} \citestyle{chicago}
% -----------------------------------------------------------------
%% tlač !!!
\usepackage[pdftex,unicode=true,bookmarksnumbered=true,
bookmarksopen=true,pdfmenubar=true,pdfview=Fit,linktocpage=true,
pageanchor=true,bookmarkstype=toc,pdfpagemode=UseOutlines,
pdfstartpage=1]{hyperref}
\hypersetup{%
baseurl={http://www.tuke.sk/sevcovic},
pdfcreator={pdfLaTeX},
pdfkeywords={Optimization, thesis, writing},
pdftitle={The Optimization of the Thesis Writing},
pdfauthor={Vojtech Riník},
pdfsubject={Bachelor's Thesis}
} 

%% -----------------------------------------------------------------
%% START YOUR THESIS
%% -----------------------------------------------------------------
%%
%% PLEASE SELECT YOUR PREFERED THESIS TYPE
%%
%% A Bachelor's degree is a first degree at college or university
\bachelorthesis{Appendix B: System guide}
%%
%% A Master's thesis is a second level college or university degree
%\masterthesis{Master's Thesis}
%% -----------------------------------------------------------------
%% Ak praca nema 'podnazov' zakomentujte riadky \subtitle a \podnazov, 
%% alebo polozky nechajte prazdne
\author{Vojtech Riník}
\title{Atmosphere: Concurrency enabled data synchronization platform with HTML5/JS and Cocoa clients}
\subtitle{}
\abstrakte{This thesis describes Atmosphere, a platform that improves user experience of web, mobile and desktop applications by storing remote objects in local database. The first chapters describe the background and state of existing solutions. Next chapters focus on available technologies that could be employed. With this knowledge the thesis proposes a design for the platform. The implementations are described in detail in the next chapter including example applications. The thesis is concluded by comparing Atmosphere to other projects.}
\keywords{Web applications, Cocoa, Synchronization, Real-time applications}
\degree{Bachelor}
\university{Technical University of Košice}
\faculty{Faculty of Electrical Engineering and Informatics}
\facultyabbr{FEI}
\department{Department of Computers and Informatics}
\departmentabbr{KPI}
\fieldofstudy{9.2.1 Informatics}
\studyprogramme{Informatics}
\supervisor{Assoc. Prof. Ing. František Jakab, PhD.}
\consultanta{Ing. Ivan Klimek}
% \consultantb{RNDr.~Marián Čierny, DrSc.}
\dateofsubmission{May. 22. 2012} 
\town{Košice}
\abstrakt{Táto práca opisuje Atmosphere, platformu, ktorá zlepšuje používanie internetových aplikácii tým, že vzdialené objekty ukladá v lokálnej pamäti. Prvé kapitoly opisuju pozadie a stav terajších riešení. Ďalšie kapitoly sú zamerané na dostupné technológie, ktoré by mohli byť využité. S týmito znalostiami práca navrhuje štruktúru platformy. Implementácie a ukážkové aplikácie sú detailne popísané v ďalších kapitolách. Práca je zakončená porovnaním Atmosphere a iných projektov.}
\klucoveslova{Webové aplikácie, Cocoa, Synchronizácia, Aplikácie v reálnom čase}

\begin{document}
\renewcommand\theHfigure{\theHsection.\arabic{figure}}
\renewcommand\theHtable{\theHsection.\arabic{table}}
\bibliographystyle{dcu}
%% input the 'First page of the Thesis'
\firstpage

%% input the 'Title page of the Thesis'
\titlepage

%\addcontentsline{toc}{section}{\numberline{}List of Therms}

% \setlength{\parindent}{1cm} 
% \setlength{\parskip}{0cm}

%
%
\renewcommand{\thesection} {\Alph{section}}
\setcounter{section}{1}
\section{System guide}

\subsection{JavaScript library}

All of the following sources are available in the \href{https://github.com/vojto/atmos2}{GitHub repository of Atmosphere}.

\subsubsection{AppContext}

\subsubsection*{Pubic Methods}
\item
\textbf{exists}
Check if object exists.
\item
\textbf{updateOrCreate}
Updates or creates object with passed data.
\item
\textbf{create}
Creates new object.
\item
\textbf{update}
Updates object with data.
\item
\textbf{changeID}
Changes ID of an object.
\item
\textbf{relation}
Sets up relationship for an object.
\item
\textbf{objectAtURI}
Returns object at passed URI.
\item
\textbf{dataForURI}
Returns data for object at passed URI.
\item
\textbf{dataForObject}
Returns data for object.
\item
\textbf{objectURI}
Returns URI of an object.
\item
\textbf{allURIs}
Returns URIs of all objeccts.
\item
\textbf{destroy}
Destroy object.
\subsubsection*{Private Methods}
\item
\textbf{\_modelForURI}
Returns model for URI.

\subsubsection{MessageClient}
\subsubsection*{Pubic Methods}
\item
\textbf{connect}
Connects the client.
\item
\textbf{close}
Closes the connection.
\item
\textbf{socketDidClose}
Callback called when socket closes.
\item
\textbf{send}
Sends a message.
\item
\textbf{parseNotification}
Parses incoming notification.
\item
\textbf{parseUpdate}
Parses incoming update notification

\subsubsection{MetaContext}
\subsubsection*{Pubic Methods}
\item
\textbf{configure}
Configures meta context.
\item
\textbf{markURIChanged}
Marks object at URI as changed.
\item
\textbf{findOrCreateObjectAtURI}
Finds or creates meta object for URI.
\item
\textbf{objectAtURI}
Returns meta object for URI.
\item
\textbf{createObjectAtURI}
Creates new meta object for URI.
\item
\textbf{saveObject}
Saves meta object.
\item
\textbf{deleteObject}
Deletes meta object.
\item
\textbf{changeIDAtURI}
Changes ID of meta object for URI.
\item
\textbf{isURILocalOnly}
Returns true if object at URI has flag of being local only.
\item
\textbf{isURIChanged}
Returns true if object at URI has flag of being changed.
\item
\textbf{changedObjects}
Returns all changed objects.
\item
\textbf{markURISynced}
Marks object at URI as synced. (Removes the flag of being changed.)

\subsubsection{ResourceClient}
\subsubsection*{Pubic Methods}
\item
\textbf{fetch}
Fetches remote items.
\item
\textbf{updateFromItems}
Updates local storage from remote JSON object.
\item
\textbf{updateFromItem}
Updates local object from remote JSON object.
\item
\textbf{itemsFromResult}
Extracts items from JSON result.
\item
\textbf{save}
Saves object remotely.
\item
\textbf{execute}
Executes requests against remote service.
\item
\textbf{request}
Creates remote request.
\item
\textbf{ajax}
Makes AJAX call.
\item
\textbf{addHeader}
Adds header to be sent with every HTTP request.

\subsubsection*{Private Methods}
\item
\textbf{\_updateFromData}
Updates local object from data.
\item
\textbf{\_removeObjectsNotInList}
Removes local objects that were not included in the response.
\item
\textbf{\_findPathForObject}
Returns remote path for local object.
\item
\textbf{\_findPathForURI}
Returns remote path for object with URI.
\item
\textbf{\_findPath}
Finds path using routing table.

\subsubsection{Synchronizer}
\subsubsection*{Pubic Methods}

\item
\textbf{constructor}
Constructs synchronizer.
\item
\textbf{bind}
Binds event.
\item
\textbf{trigger}
Triggers event.
\item
\textbf{unbind}
Unbinds event.
\item
\textbf{updateOrCreate}
Updates or create local object with data specified.
\item
\textbf{fetch}
Fetches remote object.
\item
\textbf{save}
Saves local object remotely.
\item
\textbf{execute}
Executes request.
\item
\textbf{request}
Creates request.
\item
\textbf{markObjectChanged}
Marks object as changed.
\item
\textbf{markURISynced}
Marks object as synced.
\item
\textbf{setNeedsSync}
Schedules synchronization for the next cycle.
\item
\textbf{startSync}
Starts synchronization.
\item
\textbf{removeObjectsNotInList}
Removes objects not included in the list passed.
\item
\textbf{setAuthKey}
Sets authentication key.
\item
\textbf{hasAuthKey}
Checks if authentication key is present.
\item
\textbf{didAuth}
Callback for successful authentication.
\item
\textbf{didFailAuth}
Callback for failed authentication.


\subsection{Cocoa library}

The source code for Cocoa library is not included here due to its size. It can be found in \href{https://github.com/vojto/atmos2-cocoa}{GitHub repository of Atmosphere}.

Here follows the documentation for main classes and data structures of Cocoa implementation.

\hypertarget{struct___a_t_object_u_r_i}{
\subsubsection{\_\-ATObjectURI Struct Reference}
\label{struct___a_t_object_u_r_i}\index{\_\-ATObjectURI@{\_\-ATObjectURI}}
}
\subsubsection*{Public Attributes}
\begin{DoxyCompactItemize}
\item 
\hypertarget{struct___a_t_object_u_r_i_a868f797198138714f1e35d12a2bbf204}{
\hyperlink{class_n_s_string}{NSString} $\ast$ {\bfseries entity}}
\label{struct___a_t_object_u_r_i_a868f797198138714f1e35d12a2bbf204}

\item 
\hypertarget{struct___a_t_object_u_r_i_a3a07f6f3a3a48e6a0802f6cc0c3b6fac}{
\hyperlink{class_n_s_string}{NSString} $\ast$ {\bfseries identifier}}
\label{struct___a_t_object_u_r_i_a3a07f6f3a3a48e6a0802f6cc0c3b6fac}

\end{DoxyCompactItemize}


The documentation for this struct was generated from the following file:\begin{DoxyCompactItemize}
\item 
ATObjectURI.h\end{DoxyCompactItemize}

\hypertarget{struct___a_t_route}{
\subsubsection{\_\-ATRoute Struct Reference}
\label{struct___a_t_route}\index{\_\-ATRoute@{\_\-ATRoute}}
}
\subsubsection*{Public Attributes}
\begin{DoxyCompactItemize}
\item 
\hypertarget{struct___a_t_route_ad49d79eead75bfa2643701487f21e467}{
RKRequestMethod {\bfseries method}}
\label{struct___a_t_route_ad49d79eead75bfa2643701487f21e467}

\item 
\hypertarget{struct___a_t_route_a5286920c5fea2d64c8cb39fecdfccfe8}{
\hyperlink{class_n_s_string}{NSString} $\ast$ {\bfseries path}}
\label{struct___a_t_route_a5286920c5fea2d64c8cb39fecdfccfe8}

\end{DoxyCompactItemize}


The documentation for this struct was generated from the following file:\begin{DoxyCompactItemize}
\item 
ATResourceClient.h\end{DoxyCompactItemize}

\hypertarget{interface_a_t_app_context}{
\subsubsection{ATAppContext Class Reference}
\label{interface_a_t_app_context}\index{ATAppContext@{ATAppContext}}
}
Inheritance diagram for ATAppContext:\begin{figure}[h]
\begin{center}
\leavevmode
\includegraphics[height=2.000000cm]{interface_a_t_app_context}
\end{center}
\end{figure}
\subsubsection*{Public Member Functions}
\begin{DoxyCompactItemize}
\item 
\hypertarget{interface_a_t_app_context_a7e4fa14a4fba295c242147d0833ebfed}{
(id) -\/ {\bfseries initWithSynchronizer:appContext:}}
\label{interface_a_t_app_context_a7e4fa14a4fba295c242147d0833ebfed}

\item 
\hypertarget{interface_a_t_app_context_a94a02a8d89c8ae918d7ddb36ba527b48}{
(\hyperlink{class_n_s_managed_object}{NSManagedObject} $\ast$) -\/ {\bfseries objectAtURI:}}
\label{interface_a_t_app_context_a94a02a8d89c8ae918d7ddb36ba527b48}

\item 
\hypertarget{interface_a_t_app_context_a21b9499e171d899e7aecb70b582182ba}{
(\hyperlink{class_n_s_managed_object}{NSManagedObject} $\ast$) -\/ {\bfseries createAppObjectAtURI:}}
\label{interface_a_t_app_context_a21b9499e171d899e7aecb70b582182ba}

\item 
\hypertarget{interface_a_t_app_context_a3df88879a1cacecd35a7bc674f8fe605}{
(Class) -\/ {\bfseries \_\-managedClassForURI:}}
\label{interface_a_t_app_context_a3df88879a1cacecd35a7bc674f8fe605}

\item 
\hypertarget{interface_a_t_app_context_af5b706906b12bbae130ba97a1fd4f233}{
(\hyperlink{struct___a_t_object_u_r_i}{ATObjectURI}) -\/ {\bfseries URIOfAppObject:}}
\label{interface_a_t_app_context_af5b706906b12bbae130ba97a1fd4f233}

\item 
\hypertarget{interface_a_t_app_context_a5467ab26d0768fda050e72b7267dd4f2}{
(void) -\/ {\bfseries changeIDTo:atURI:}}
\label{interface_a_t_app_context_a5467ab26d0768fda050e72b7267dd4f2}

\item 
\hypertarget{interface_a_t_app_context_ab4b436e64d15f8d25d3f186479e4cf7d}{
(void) -\/ {\bfseries updateAppObject:withDictionary:}}
\label{interface_a_t_app_context_ab4b436e64d15f8d25d3f186479e4cf7d}

\item 
\hypertarget{interface_a_t_app_context_a2b43c323a64dcef375733da94ee63fd2}{
(void) -\/ {\bfseries deleteAppObject:}}
\label{interface_a_t_app_context_a2b43c323a64dcef375733da94ee63fd2}

\item 
\hypertarget{interface_a_t_app_context_ad590a1e779fcebf73218573783425151}{
(void) -\/ {\bfseries \_\-resolveRelations:withDictionary:}}
\label{interface_a_t_app_context_ad590a1e779fcebf73218573783425151}

\item 
\hypertarget{interface_a_t_app_context_a544871bcc8c6974f3bdf0edce58b5468}{
(NSDictionary $\ast$) -\/ {\bfseries dataForObject:}}
\label{interface_a_t_app_context_a544871bcc8c6974f3bdf0edce58b5468}

\item 
\hypertarget{interface_a_t_app_context_ab14fb60385ad32a8a2e91fa185c1f7d4}{
(NSArray $\ast$) -\/ {\bfseries relationsForAppObject:}}
\label{interface_a_t_app_context_ab14fb60385ad32a8a2e91fa185c1f7d4}

\item 
\hypertarget{interface_a_t_app_context_a344bf2060f78ed0968a5cf22c76949d2}{
(BOOL) -\/ {\bfseries attributesChangedInAppObject:}}
\label{interface_a_t_app_context_a344bf2060f78ed0968a5cf22c76949d2}

\item 
\hypertarget{interface_a_t_app_context_aa6f9c0e4fd9b9d542eda09bfcc2c2f7a}{
(BOOL) -\/ {\bfseries hasChanges}}
\label{interface_a_t_app_context_aa6f9c0e4fd9b9d542eda09bfcc2c2f7a}

\item 
\hypertarget{interface_a_t_app_context_a00ba90e776e4df7fc7d9dc3c0137138e}{
(void) -\/ {\bfseries save}}
\label{interface_a_t_app_context_a00ba90e776e4df7fc7d9dc3c0137138e}

\item 
\hypertarget{interface_a_t_app_context_a05e2eea400233f313641bd144109e1fb}{
(void) -\/ {\bfseries save:}}
\label{interface_a_t_app_context_a05e2eea400233f313641bd144109e1fb}

\item 
\hypertarget{interface_a_t_app_context_a05643fd02b9692dfaf1ebeea5bf84a6b}{
(void) -\/ {\bfseries obtainPermanentIDsForObjects:error:}}
\label{interface_a_t_app_context_a05643fd02b9692dfaf1ebeea5bf84a6b}

\end{DoxyCompactItemize}
\subsubsection*{Static Public Member Functions}
\begin{DoxyCompactItemize}
\item 
\hypertarget{interface_a_t_app_context_a00756fedbbc0ce79dd55c20862d47913}{
(id) + {\bfseries sharedAppContext}}
\label{interface_a_t_app_context_a00756fedbbc0ce79dd55c20862d47913}

\end{DoxyCompactItemize}
\subsubsection*{Protected Attributes}
\begin{DoxyCompactItemize}
\item 
\hypertarget{interface_a_t_app_context_af1c6f509bc2183767278e1e687ffcc0e}{
\hyperlink{interface_a_t_synchronizer}{ATSynchronizer} $\ast$ {\bfseries \_\-sync}}
\label{interface_a_t_app_context_af1c6f509bc2183767278e1e687ffcc0e}

\item 
\hypertarget{interface_a_t_app_context_ade3e164d9ec290b1b752f749c84a57c1}{
NSManagedObjectContext $\ast$ {\bfseries \_\-managedContext}}
\label{interface_a_t_app_context_ade3e164d9ec290b1b752f749c84a57c1}

\item 
\hypertarget{interface_a_t_app_context_addad5fb48d0bb32ca07a120bde680072}{
NSMutableArray $\ast$ {\bfseries \_\-relationsQueue}}
\label{interface_a_t_app_context_addad5fb48d0bb32ca07a120bde680072}

\end{DoxyCompactItemize}
\subsubsection*{Properties}
\begin{DoxyCompactItemize}
\item 
\hypertarget{interface_a_t_app_context_a9e4d6b12a7d2db4e642504ec45744a85}{
\hyperlink{interface_a_t_synchronizer}{ATSynchronizer} $\ast$ {\bfseries sync}}
\label{interface_a_t_app_context_a9e4d6b12a7d2db4e642504ec45744a85}

\item 
\hypertarget{interface_a_t_app_context_a38c53145eccd1f4fbbbea4a081dfd81e}{
NSManagedObjectContext $\ast$ {\bfseries managedContext}}
\label{interface_a_t_app_context_a38c53145eccd1f4fbbbea4a081dfd81e}

\item 
\hypertarget{interface_a_t_app_context_adc64825f25a7e7c6695ddcc8cc265822}{
\hyperlink{interface_a_t_attribute_mapper}{ATAttributeMapper} $\ast$ {\bfseries attributeMapper}}
\label{interface_a_t_app_context_adc64825f25a7e7c6695ddcc8cc265822}

\end{DoxyCompactItemize}


The documentation for this class was generated from the following files:\begin{DoxyCompactItemize}
\item 
ATAppContext.h\item 
ATAppContext.m\end{DoxyCompactItemize}

\hypertarget{interface_a_t_attribute_mapper}{
\subsubsection{ATAttributeMapper Class Reference}
\label{interface_a_t_attribute_mapper}\index{ATAttributeMapper@{ATAttributeMapper}}
}


{\ttfamily \#import $<$ATAttributeMapper.h$>$}

Inheritance diagram for ATAttributeMapper:\begin{figure}[h]
\begin{center}
\leavevmode
\includegraphics[height=2.000000cm]{interface_a_t_attribute_mapper}
\end{center}
\end{figure}
\subsubsection*{Public Member Functions}
\begin{DoxyCompactItemize}
\item 
\hypertarget{interface_a_t_attribute_mapper_ae1724904b0f49669626c87f07b4e0fd2}{
(id) -\/ {\bfseries initWithMappingHelper:}}
\label{interface_a_t_attribute_mapper_ae1724904b0f49669626c87f07b4e0fd2}

\end{DoxyCompactItemize}
\subsubsection*{Properties}
\begin{DoxyCompactItemize}
\item 
\hypertarget{interface_a_t_attribute_mapper_a2b8e7175248e56bbe8cd2217e23e9220}{
\hyperlink{interface_a_t_mapping_helper}{ATMappingHelper} $\ast$ {\bfseries mappingHelper}}
\label{interface_a_t_attribute_mapper_a2b8e7175248e56bbe8cd2217e23e9220}

\end{DoxyCompactItemize}


\subsubsection{Detailed Description}
This class does nothing 

The documentation for this class was generated from the following files:\begin{DoxyCompactItemize}
\item 
ATAttributeMapper.h\item 
ATAttributeMapper.m\end{DoxyCompactItemize}

\hypertarget{interface_a_t_connection_guard}{
\subsubsection{ATConnectionGuard Class Reference}
\label{interface_a_t_connection_guard}\index{ATConnectionGuard@{ATConnectionGuard}}
}
Inheritance diagram for ATConnectionGuard:\begin{figure}[h]
\begin{center}
\leavevmode
\includegraphics[height=2.000000cm]{interface_a_t_connection_guard}
\end{center}
\end{figure}
\subsubsection*{Public Member Functions}
\begin{DoxyCompactItemize}
\item 
\hypertarget{interface_a_t_connection_guard_a96b67e50d661e8e0edf6dd280dd03a7d}{
(void) -\/ {\bfseries start}}
\label{interface_a_t_connection_guard_a96b67e50d661e8e0edf6dd280dd03a7d}

\item 
\hypertarget{interface_a_t_connection_guard_a690563a6d28297f42209c29e016a961b}{
(void) -\/ {\bfseries stop}}
\label{interface_a_t_connection_guard_a690563a6d28297f42209c29e016a961b}

\item 
\hypertarget{interface_a_t_connection_guard_adc0892e3439bd48580bc8171d3cd9c5c}{
(void) -\/ {\bfseries \_\-checkConnection}}
\label{interface_a_t_connection_guard_adc0892e3439bd48580bc8171d3cd9c5c}

\end{DoxyCompactItemize}
\subsubsection*{Protected Attributes}
\begin{DoxyCompactItemize}
\item 
\hypertarget{interface_a_t_connection_guard_af213d51050cb2c17a7f85715fa602323}{
BOOL {\bfseries isRunning}}
\label{interface_a_t_connection_guard_af213d51050cb2c17a7f85715fa602323}

\end{DoxyCompactItemize}
\subsubsection*{Properties}
\begin{DoxyCompactItemize}
\item 
\hypertarget{interface_a_t_connection_guard_a6cafd79831265da735d92cec478192a0}{
\hyperlink{interface_a_t_synchronizer}{ATSynchronizer} $\ast$ {\bfseries client}}
\label{interface_a_t_connection_guard_a6cafd79831265da735d92cec478192a0}

\end{DoxyCompactItemize}


The documentation for this class was generated from the following files:\begin{DoxyCompactItemize}
\item 
ATConnectionGuard.h\item 
ATConnectionGuard.m\end{DoxyCompactItemize}

\hypertarget{interface_a_t_entity_fetch_request}{
\subsubsection{ATEntityFetchRequest Class Reference}
\label{interface_a_t_entity_fetch_request}\index{ATEntityFetchRequest@{ATEntityFetchRequest}}
}
Inheritance diagram for ATEntityFetchRequest:\begin{figure}[h]
\begin{center}
\leavevmode
\includegraphics[height=2.000000cm]{interface_a_t_entity_fetch_request}
\end{center}
\end{figure}
\subsubsection*{Public Member Functions}
\begin{DoxyCompactItemize}
\item 
\hypertarget{interface_a_t_entity_fetch_request_a675e97f5dd42dc681c481648ef1d39b5}{
(id) -\/ {\bfseries initWithResourceClient:entity:}}
\label{interface_a_t_entity_fetch_request_a675e97f5dd42dc681c481648ef1d39b5}

\item 
\hypertarget{interface_a_t_entity_fetch_request_a72842e3aa37c2d86fb2f728aee1796fe}{
(void) -\/ {\bfseries send}}
\label{interface_a_t_entity_fetch_request_a72842e3aa37c2d86fb2f728aee1796fe}

\item 
\hypertarget{interface_a_t_entity_fetch_request_aef0c30154968bce6a9e0ead2c0c0fe98}{
(\hyperlink{struct___a_t_object_u_r_i}{ATObjectURI}) -\/ {\bfseries objectURIFromItem:}}
\label{interface_a_t_entity_fetch_request_aef0c30154968bce6a9e0ead2c0c0fe98}

\end{DoxyCompactItemize}
\subsubsection*{Properties}
\begin{DoxyCompactItemize}
\item 
\hypertarget{interface_a_t_entity_fetch_request_a187e3721b09b66577b6d2feba7ec0a4a}{
\hyperlink{interface_a_t_resource_client}{ATResourceClient} $\ast$ {\bfseries resourceClient}}
\label{interface_a_t_entity_fetch_request_a187e3721b09b66577b6d2feba7ec0a4a}

\item 
\hypertarget{interface_a_t_entity_fetch_request_ab59eb37b5bd6a386573c0ef591cad967}{
RKClient $\ast$ {\bfseries networkClient}}
\label{interface_a_t_entity_fetch_request_ab59eb37b5bd6a386573c0ef591cad967}

\item 
\hypertarget{interface_a_t_entity_fetch_request_afab560a9170c08b33c098f15b4399c01}{
\hyperlink{class_n_s_string}{NSString} $\ast$ {\bfseries entity}}
\label{interface_a_t_entity_fetch_request_afab560a9170c08b33c098f15b4399c01}

\end{DoxyCompactItemize}


The documentation for this class was generated from the following files:\begin{DoxyCompactItemize}
\item 
ATEntityFetchRequest.h\item 
ATEntityFetchRequest.m\end{DoxyCompactItemize}

\hypertarget{interface_a_t_mapping_helper}{
\subsubsection{ATMappingHelper Class Reference}
\label{interface_a_t_mapping_helper}\index{ATMappingHelper@{ATMappingHelper}}
}
Inheritance diagram for ATMappingHelper:\begin{figure}[h]
\begin{center}
\leavevmode
\includegraphics[height=2.000000cm]{interface_a_t_mapping_helper}
\end{center}
\end{figure}
\subsubsection*{Public Member Functions}
\begin{DoxyCompactItemize}
\item 
\hypertarget{interface_a_t_mapping_helper_a73fc4c531c66209f448ed0d79835fe56}{
(void) -\/ {\bfseries loadEntitiesMapFromResource:}}
\label{interface_a_t_mapping_helper_a73fc4c531c66209f448ed0d79835fe56}

\item 
\hypertarget{interface_a_t_mapping_helper_a9ec9ef5dc665692dce4fca0e6d050a0e}{
(void) -\/ {\bfseries loadAttributesMapFromResource:}}
\label{interface_a_t_mapping_helper_a9ec9ef5dc665692dce4fca0e6d050a0e}

\item 
\hypertarget{interface_a_t_mapping_helper_a9c616f062a025869711abbfe94796581}{
(void) -\/ {\bfseries loadRelationsMapFromResource:}}
\label{interface_a_t_mapping_helper_a9c616f062a025869711abbfe94796581}

\item 
\hypertarget{interface_a_t_mapping_helper_a0ab20c0f77c385dac21377dee445a3f6}{
(void) -\/ {\bfseries \_\-loadResource:intoDictionary:}}
\label{interface_a_t_mapping_helper_a0ab20c0f77c385dac21377dee445a3f6}

\item 
\hypertarget{interface_a_t_mapping_helper_a9fa5dd9036bd1e47208ca98a11003b2f}{
(\hyperlink{class_n_s_string}{NSString} $\ast$) -\/ {\bfseries localEntityNameFor:}}
\label{interface_a_t_mapping_helper_a9fa5dd9036bd1e47208ca98a11003b2f}

\item 
\hypertarget{interface_a_t_mapping_helper_aea4cc83c108b4dce637d7188b2015e5d}{
(\hyperlink{class_n_s_string}{NSString} $\ast$) -\/ {\bfseries serverEntityNameFor:}}
\label{interface_a_t_mapping_helper_aea4cc83c108b4dce637d7188b2015e5d}

\item 
\hypertarget{interface_a_t_mapping_helper_a339e9e1ee190e95be0c00cc86dc3794c}{
(\hyperlink{class_n_s_string}{NSString} $\ast$) -\/ {\bfseries serverEntityNameForAppObject:}}
\label{interface_a_t_mapping_helper_a339e9e1ee190e95be0c00cc86dc3794c}

\item 
\hypertarget{interface_a_t_mapping_helper_a045ecbcd62366a3de84fccc190bc7697}{
(\hyperlink{class_n_s_string}{NSString} $\ast$) -\/ {\bfseries serverAttributeNameFor:entity:}}
\label{interface_a_t_mapping_helper_a045ecbcd62366a3de84fccc190bc7697}

\item 
\hypertarget{interface_a_t_mapping_helper_a6096407dd2fee1da8af9a7dfc3256219}{
(\hyperlink{class_n_s_string}{NSString} $\ast$) -\/ {\bfseries localAttributeNameFor:entity:}}
\label{interface_a_t_mapping_helper_a6096407dd2fee1da8af9a7dfc3256219}

\item 
\hypertarget{interface_a_t_mapping_helper_af0532a90720b80d8c68f569cbef709c2}{
(NSDictionary $\ast$) -\/ {\bfseries relationsForObject:}}
\label{interface_a_t_mapping_helper_af0532a90720b80d8c68f569cbef709c2}

\item 
\hypertarget{interface_a_t_mapping_helper_aebeeba57781626e2ce67e7cdc695ba41}{
(NSDictionary $\ast$) -\/ {\bfseries relationsForEntity:}}
\label{interface_a_t_mapping_helper_aebeeba57781626e2ce67e7cdc695ba41}

\end{DoxyCompactItemize}
\subsubsection*{Properties}
\begin{DoxyCompactItemize}
\item 
NSDictionary $\ast$ \hyperlink{interface_a_t_mapping_helper_af42b5b8a8f037917548813611383da0f}{entitiesMap}
\item 
\hypertarget{interface_a_t_mapping_helper_abd9d756b7b5ed7a788ffcd9d1c8acc5c}{
NSDictionary $\ast$ {\bfseries attributesMap}}
\label{interface_a_t_mapping_helper_abd9d756b7b5ed7a788ffcd9d1c8acc5c}

\item 
\hypertarget{interface_a_t_mapping_helper_a9545ea64a1c4a6564aba6d30155fd47f}{
NSDictionary $\ast$ {\bfseries relationsMap}}
\label{interface_a_t_mapping_helper_a9545ea64a1c4a6564aba6d30155fd47f}

\end{DoxyCompactItemize}


\subsubsection{Property Documentation}
\hypertarget{interface_a_t_mapping_helper_af42b5b8a8f037917548813611383da0f}{
\index{ATMappingHelper@{ATMappingHelper}!entitiesMap@{entitiesMap}}
\index{entitiesMap@{entitiesMap}!ATMappingHelper@{ATMappingHelper}}
\subsubsection[{entitiesMap}]{\setlength{\rightskip}{0pt plus 5cm}-\/ (NSDictionary$\ast$) entitiesMap\hspace{0.3cm}{\ttfamily  \mbox{[}read, write, retain\mbox{]}}}}
\label{interface_a_t_mapping_helper_af42b5b8a8f037917548813611383da0f}
Entities map dictionary. Key represents server entity name and value represents client entity name 

The documentation for this class was generated from the following files:\begin{DoxyCompactItemize}
\item 
ATMappingHelper.h\item 
ATMappingHelper.m\end{DoxyCompactItemize}

\hypertarget{interface_a_t_message}{
\subsubsection{ATMessage Class Reference}
\label{interface_a_t_message}\index{ATMessage@{ATMessage}}
}
Inheritance diagram for ATMessage:\begin{figure}[h]
\begin{center}
\leavevmode
\includegraphics[height=2.000000cm]{interface_a_t_message}
\end{center}
\end{figure}
\subsubsection*{Public Member Functions}
\begin{DoxyCompactItemize}
\item 
\hypertarget{interface_a_t_message_a479b92f064ea3982fe3fa1dc0046da64}{
(\hyperlink{class_n_s_string}{NSString} $\ast$) -\/ {\bfseries JSONString}}
\label{interface_a_t_message_a479b92f064ea3982fe3fa1dc0046da64}

\end{DoxyCompactItemize}
\subsubsection*{Static Public Member Functions}
\begin{DoxyCompactItemize}
\item 
\hypertarget{interface_a_t_message_a899be809d07eddd49c7a69edcdbbc643}{
(\hyperlink{interface_a_t_message}{ATMessage} $\ast$) + {\bfseries messageFromJSONString:}}
\label{interface_a_t_message_a899be809d07eddd49c7a69edcdbbc643}

\end{DoxyCompactItemize}
\subsubsection*{Protected Attributes}
\begin{DoxyCompactItemize}
\item 
\hypertarget{interface_a_t_message_a55260e89bdf6b62ed070c9c3109e1717}{
\hyperlink{class_n_s_string}{NSString} $\ast$ {\bfseries \_\-type}}
\label{interface_a_t_message_a55260e89bdf6b62ed070c9c3109e1717}

\item 
\hypertarget{interface_a_t_message_adfc3efefff0092008dfc5a77ab28f093}{
NSDictionary $\ast$ {\bfseries \_\-content}}
\label{interface_a_t_message_adfc3efefff0092008dfc5a77ab28f093}

\end{DoxyCompactItemize}
\subsubsection*{Properties}
\begin{DoxyCompactItemize}
\item 
\hypertarget{interface_a_t_message_ae76aa0e54d3ec8380ba8b7d8fbcf7d9f}{
\hyperlink{class_n_s_string}{NSString} $\ast$ {\bfseries type}}
\label{interface_a_t_message_ae76aa0e54d3ec8380ba8b7d8fbcf7d9f}

\item 
\hypertarget{interface_a_t_message_a42e10d45f4235440c77a217bd6fcb2a8}{
NSDictionary $\ast$ {\bfseries content}}
\label{interface_a_t_message_a42e10d45f4235440c77a217bd6fcb2a8}

\end{DoxyCompactItemize}


The documentation for this class was generated from the following files:\begin{DoxyCompactItemize}
\item 
ATMessage.h\item 
ATMessage.m\end{DoxyCompactItemize}

\hypertarget{interface_a_t_message_client}{
\subsubsection{ATMessageClient Class Reference}
\label{interface_a_t_message_client}\index{ATMessageClient@{ATMessageClient}}
}


{\ttfamily \#import $<$ATMessageClient.h$>$}

Inheritance diagram for ATMessageClient:\begin{figure}[h]
\begin{center}
\leavevmode
\includegraphics[height=2.000000cm]{interface_a_t_message_client}
\end{center}
\end{figure}
\subsubsection*{Public Member Functions}
\begin{DoxyCompactItemize}
\item 
\hypertarget{interface_a_t_message_client_ac3819bad54449ea9f46b3d9929bbf435}{
(id) -\/ {\bfseries initWithSynchronizer:}}
\label{interface_a_t_message_client_ac3819bad54449ea9f46b3d9929bbf435}

\item 
\hypertarget{interface_a_t_message_client_a5dc5708c4e13d7a75f78050b5f15249c}{
(void) -\/ {\bfseries connect}}
\label{interface_a_t_message_client_a5dc5708c4e13d7a75f78050b5f15249c}

\item 
\hypertarget{interface_a_t_message_client_ade89bb992bb85887edb8c22f877315e3}{
(BOOL) -\/ {\bfseries isConnected}}
\label{interface_a_t_message_client_ade89bb992bb85887edb8c22f877315e3}

\item 
\hypertarget{interface_a_t_message_client_aaf252625e14f5b125bc246eaa85a0c15}{
(void) -\/ {\bfseries \_\-initializeSocketConnection}}
\label{interface_a_t_message_client_aaf252625e14f5b125bc246eaa85a0c15}

\item 
\hypertarget{interface_a_t_message_client_ac73d19a7651d8e492c9ac97325640e0a}{
(void) -\/ {\bfseries \_\-sendConnectMessage}}
\label{interface_a_t_message_client_ac73d19a7651d8e492c9ac97325640e0a}

\item 
\hypertarget{interface_a_t_message_client_a79c2b59c186f18881680ab4fc1f277d1}{
(void) -\/ {\bfseries disconnect}}
\label{interface_a_t_message_client_a79c2b59c186f18881680ab4fc1f277d1}

\item 
\hypertarget{interface_a_t_message_client_a23c6f5dbb5a6358341d1707b91b86911}{
(void) -\/ {\bfseries \_\-didReceiveServerAuthFailure:}}
\label{interface_a_t_message_client_a23c6f5dbb5a6358341d1707b91b86911}

\item 
\hypertarget{interface_a_t_message_client_a2b1b29a59e25f19e2395b8cd51cf4c26}{
(void) -\/ {\bfseries \_\-didReceiveServerAuthSuccess:}}
\label{interface_a_t_message_client_a2b1b29a59e25f19e2395b8cd51cf4c26}

\item 
\hypertarget{interface_a_t_message_client_a03ec182577d528bc36c8999ea7e5b074}{
(void) -\/ {\bfseries sendMessage:}}
\label{interface_a_t_message_client_a03ec182577d528bc36c8999ea7e5b074}

\item 
\hypertarget{interface_a_t_message_client_a8e98344567ab31d6976978d5ccc172bd}{
(void) -\/ {\bfseries \_\-didReceiveServerPush:}}
\label{interface_a_t_message_client_a8e98344567ab31d6976978d5ccc172bd}

\end{DoxyCompactItemize}
\subsubsection*{Protected Attributes}
\begin{DoxyCompactItemize}
\item 
\hypertarget{interface_a_t_message_client_a59f618866758b87a03b2814d597ea143}{
BOOL {\bfseries \_\-isRunning}}
\label{interface_a_t_message_client_a59f618866758b87a03b2814d597ea143}

\item 
\hypertarget{interface_a_t_message_client_acd349f1b49d1ce1f335c58816bb8219a}{
\hyperlink{interface_a_t_synchronizer}{ATSynchronizer} $\ast$ {\bfseries \_\-sync}}
\label{interface_a_t_message_client_acd349f1b49d1ce1f335c58816bb8219a}

\item 
\hyperlink{class_n_s_string}{NSString} $\ast$ \hyperlink{interface_a_t_message_client_a10e7fe2e07ea9c28ca3cc987f85bff10}{\_\-host}
\item 
\hypertarget{interface_a_t_message_client_a13e73da308c7e71f864516a7b818d06c}{
NSInteger {\bfseries \_\-port}}
\label{interface_a_t_message_client_a13e73da308c7e71f864516a7b818d06c}

\item 
\hypertarget{interface_a_t_message_client_ad7ddc109d962798a2e64d7ba292e19e4}{
SocketIO $\ast$ {\bfseries \_\-connection}}
\label{interface_a_t_message_client_ad7ddc109d962798a2e64d7ba292e19e4}

\end{DoxyCompactItemize}
\subsubsection*{Properties}
\begin{DoxyCompactItemize}
\item 
\hypertarget{interface_a_t_message_client_a32d4d7b9e4f15f1595efca5ab85b1381}{
\hyperlink{interface_a_t_synchronizer}{ATSynchronizer} $\ast$ {\bfseries sync}}
\label{interface_a_t_message_client_a32d4d7b9e4f15f1595efca5ab85b1381}

\item 
\hypertarget{interface_a_t_message_client_a330b8c86d98c6ef9cbd8a8f3e2cae457}{
\hyperlink{class_n_s_string}{NSString} $\ast$ {\bfseries host}}
\label{interface_a_t_message_client_a330b8c86d98c6ef9cbd8a8f3e2cae457}

\item 
\hypertarget{interface_a_t_message_client_a8b889518a76ec8f627e401188761ebf1}{
NSInteger {\bfseries port}}
\label{interface_a_t_message_client_a8b889518a76ec8f627e401188761ebf1}

\item 
\hypertarget{interface_a_t_message_client_a301eb41f17a8ab3300e3d84b06ee8ceb}{
SocketIO $\ast$ {\bfseries connection}}
\label{interface_a_t_message_client_a301eb41f17a8ab3300e3d84b06ee8ceb}

\end{DoxyCompactItemize}


\subsubsection{Detailed Description}
\hyperlink{interface_a_t_message_client}{ATMessageClient} is responsible for dealing with live connection for push updates and notifications. 

\subsubsection{Member Data Documentation}
\hypertarget{interface_a_t_message_client_a10e7fe2e07ea9c28ca3cc987f85bff10}{
\index{ATMessageClient@{ATMessageClient}!\_\-host@{\_\-host}}
\index{\_\-host@{\_\-host}!ATMessageClient@{ATMessageClient}}
\subsubsection[{\_\-host}]{\setlength{\rightskip}{0pt plus 5cm}-\/ ({\bf NSString}$\ast$) {\bf \_\-host}\hspace{0.3cm}{\ttfamily  \mbox{[}protected\mbox{]}}}}
\label{interface_a_t_message_client_a10e7fe2e07ea9c28ca3cc987f85bff10}
Connection 

The documentation for this class was generated from the following files:\begin{DoxyCompactItemize}
\item 
ATMessageClient.h\item 
ATMessageClient.m\end{DoxyCompactItemize}

\hypertarget{interface_a_t_meta_context}{
\subsubsection{ATMetaContext Class Reference}
\label{interface_a_t_meta_context}\index{ATMetaContext@{ATMetaContext}}
}
Inheritance diagram for ATMetaContext:\begin{figure}[h]
\begin{center}
\leavevmode
\includegraphics[height=2.000000cm]{interface_a_t_meta_context}
\end{center}
\end{figure}
\subsubsection*{Public Member Functions}
\begin{DoxyCompactItemize}
\item 
\hypertarget{interface_a_t_meta_context_adfa6a19a0196a1e47d82781f888c6136}{
(BOOL) -\/ {\bfseries save}}
\label{interface_a_t_meta_context_adfa6a19a0196a1e47d82781f888c6136}

\item 
\hypertarget{interface_a_t_meta_context_a374a353d6271cd493f9c5cf33630eb6b}{
(void) -\/ {\bfseries markURIChanged:}}
\label{interface_a_t_meta_context_a374a353d6271cd493f9c5cf33630eb6b}

\item 
\hypertarget{interface_a_t_meta_context_a6bcf5e7ee7c5b9a7c0fbccdf7723623c}{
(void) -\/ {\bfseries markURISynced:}}
\label{interface_a_t_meta_context_a6bcf5e7ee7c5b9a7c0fbccdf7723623c}

\item 
\hypertarget{interface_a_t_meta_context_a206094bc4db6c78668fb4593f0d59fec}{
(\hyperlink{interface_a_t_meta_object}{ATMetaObject} $\ast$) -\/ {\bfseries objectAtURI:}}
\label{interface_a_t_meta_context_a206094bc4db6c78668fb4593f0d59fec}

\item 
\hypertarget{interface_a_t_meta_context_a7b970df053c1bc1869ccb99945f383fb}{
(\hyperlink{interface_a_t_meta_object}{ATMetaObject} $\ast$) -\/ {\bfseries ensureObjectAtURI:}}
\label{interface_a_t_meta_context_a7b970df053c1bc1869ccb99945f383fb}

\item 
\hypertarget{interface_a_t_meta_context_af23b62b46612b08980a321f0b23cedd4}{
(\hyperlink{interface_a_t_meta_object}{ATMetaObject} $\ast$) -\/ {\bfseries createObjectAtURI:}}
\label{interface_a_t_meta_context_af23b62b46612b08980a321f0b23cedd4}

\item 
\hypertarget{interface_a_t_meta_context_a0cd0fe9a4ed8999db95f9bd843fb6536}{
(NSArray $\ast$) -\/ {\bfseries changedObjects}}
\label{interface_a_t_meta_context_a0cd0fe9a4ed8999db95f9bd843fb6536}

\item 
\hypertarget{interface_a_t_meta_context_a154ebbae634a69798821aa99f63e36f6}{
(void) -\/ {\bfseries changeIDTo:atURI:}}
\label{interface_a_t_meta_context_a154ebbae634a69798821aa99f63e36f6}

\end{DoxyCompactItemize}
\subsubsection*{Static Public Member Functions}
\begin{DoxyCompactItemize}
\item 
\hypertarget{interface_a_t_meta_context_a5ea6f84cd61278e7a69a5e466acb5ef6}{
(id) + {\bfseries restore}}
\label{interface_a_t_meta_context_a5ea6f84cd61278e7a69a5e466acb5ef6}

\item 
\hypertarget{interface_a_t_meta_context_aa958a238c2f8e661d92410aaa365af25}{
(\hyperlink{class_n_s_string}{NSString} $\ast$) + {\bfseries path}}
\label{interface_a_t_meta_context_aa958a238c2f8e661d92410aaa365af25}

\end{DoxyCompactItemize}
\subsubsection*{Protected Attributes}
\begin{DoxyCompactItemize}
\item 
\hypertarget{interface_a_t_meta_context_aa7bba9acdaedcefd8aeb096a36181a94}{
NSMutableDictionary $\ast$ {\bfseries \_\-objects}}
\label{interface_a_t_meta_context_aa7bba9acdaedcefd8aeb096a36181a94}

\end{DoxyCompactItemize}


The documentation for this class was generated from the following files:\begin{DoxyCompactItemize}
\item 
ATMetaContext.h\item 
ATMetaContext.m\end{DoxyCompactItemize}

\hypertarget{interface_a_t_meta_object}{
\subsubsection{ATMetaObject Class Reference}
\label{interface_a_t_meta_object}\index{ATMetaObject@{ATMetaObject}}
}
Inheritance diagram for ATMetaObject:\begin{figure}[h]
\begin{center}
\leavevmode
\includegraphics[height=2.000000cm]{interface_a_t_meta_object}
\end{center}
\end{figure}
\subsubsection*{Public Member Functions}
\begin{DoxyCompactItemize}
\item 
\hypertarget{interface_a_t_meta_object_afa20cebbc71d616926540f55230ed5ab}{
(id) -\/ {\bfseries initWithURI:}}
\label{interface_a_t_meta_object_afa20cebbc71d616926540f55230ed5ab}

\end{DoxyCompactItemize}
\subsubsection*{Properties}
\begin{DoxyCompactItemize}
\item 
\hypertarget{interface_a_t_meta_object_a7ecba236000b74241ae3931ccfc3e149}{
\hyperlink{struct___a_t_object_u_r_i}{ATObjectURI} {\bfseries uri}}
\label{interface_a_t_meta_object_a7ecba236000b74241ae3931ccfc3e149}

\item 
\hypertarget{interface_a_t_meta_object_a45ed797e32382c7b79b99aba02c83ed1}{
BOOL {\bfseries isChanged}}
\label{interface_a_t_meta_object_a45ed797e32382c7b79b99aba02c83ed1}

\item 
\hypertarget{interface_a_t_meta_object_ae63d1be418eb4fc20c2ac9e2996ad963}{
BOOL {\bfseries isLocalOnly}}
\label{interface_a_t_meta_object_ae63d1be418eb4fc20c2ac9e2996ad963}

\end{DoxyCompactItemize}


The documentation for this class was generated from the following files:\begin{DoxyCompactItemize}
\item 
ATMetaObject.h\item 
ATMetaObject.m\end{DoxyCompactItemize}

\hypertarget{interface_a_t_object_save_request}{
\subsubsection{ATObjectSaveRequest Class Reference}
\label{interface_a_t_object_save_request}\index{ATObjectSaveRequest@{ATObjectSaveRequest}}
}
Inheritance diagram for ATObjectSaveRequest:\begin{figure}[h]
\begin{center}
\leavevmode
\includegraphics[height=2.000000cm]{interface_a_t_object_save_request}
\end{center}
\end{figure}
\subsubsection*{Public Member Functions}
\begin{DoxyCompactItemize}
\item 
\hypertarget{interface_a_t_object_save_request_a4251843e836ae9afa7b5ffb5e30bdb0f}{
(id) -\/ {\bfseries initWithResourceClient:object:options:}}
\label{interface_a_t_object_save_request_a4251843e836ae9afa7b5ffb5e30bdb0f}

\item 
\hypertarget{interface_a_t_object_save_request_a8bd540fab41eb1cbff19cba6519bf0f6}{
(void) -\/ {\bfseries send}}
\label{interface_a_t_object_save_request_a8bd540fab41eb1cbff19cba6519bf0f6}

\end{DoxyCompactItemize}
\subsubsection*{Properties}
\begin{DoxyCompactItemize}
\item 
\hypertarget{interface_a_t_object_save_request_a2d8d291a3e16bcea48ab0675b50e7cc6}{
\hyperlink{interface_a_t_resource_client}{ATResourceClient} $\ast$ {\bfseries resourceClient}}
\label{interface_a_t_object_save_request_a2d8d291a3e16bcea48ab0675b50e7cc6}

\item 
\hypertarget{interface_a_t_object_save_request_aec0386776e46384a4284711048edc747}{
RKClient $\ast$ {\bfseries networkClient}}
\label{interface_a_t_object_save_request_aec0386776e46384a4284711048edc747}

\item 
\hypertarget{interface_a_t_object_save_request_ac8448ee55f9108a6627ae8942be5ac84}{
NSDictionary $\ast$ {\bfseries options}}
\label{interface_a_t_object_save_request_ac8448ee55f9108a6627ae8942be5ac84}

\item 
\hypertarget{interface_a_t_object_save_request_a06cf1afe4dfd9bd6f851ed62175a4b2b}{
\hyperlink{class_n_s_managed_object}{NSManagedObject} $\ast$ {\bfseries object}}
\label{interface_a_t_object_save_request_a06cf1afe4dfd9bd6f851ed62175a4b2b}

\end{DoxyCompactItemize}


The documentation for this class was generated from the following files:\begin{DoxyCompactItemize}
\item 
ATObjectSaveRequest.h\item 
ATObjectSaveRequest.m\end{DoxyCompactItemize}

\hypertarget{interface_a_t_resource_client}{
\subsubsection{ATResourceClient Class Reference}
\label{interface_a_t_resource_client}\index{ATResourceClient@{ATResourceClient}}
}
Inheritance diagram for ATResourceClient:\begin{figure}[h]
\begin{center}
\leavevmode
\includegraphics[height=2.000000cm]{interface_a_t_resource_client}
\end{center}
\end{figure}
\subsubsection*{Public Member Functions}
\begin{DoxyCompactItemize}
\item 
\hypertarget{interface_a_t_resource_client_a6a445159cd20e385d4088227ddbb5fa7}{
(id) -\/ {\bfseries initWithSynchronizer:}}
\label{interface_a_t_resource_client_a6a445159cd20e385d4088227ddbb5fa7}

\item 
\hypertarget{interface_a_t_resource_client_acc6bee940b9654a7625fd26f9dafc0a6}{
(void) -\/ {\bfseries setBaseURL:}}
\label{interface_a_t_resource_client_acc6bee940b9654a7625fd26f9dafc0a6}

\item 
\hypertarget{interface_a_t_resource_client_a5e92652d2cd5f04296272b9ef5a69aeb}{
(void) -\/ {\bfseries addHeader:withValue:}}
\label{interface_a_t_resource_client_a5e92652d2cd5f04296272b9ef5a69aeb}

\item 
\hypertarget{interface_a_t_resource_client_a71bad71ea56242fd19e5f48784689b5c}{
(void) -\/ {\bfseries fetchEntity:}}
\label{interface_a_t_resource_client_a71bad71ea56242fd19e5f48784689b5c}

\item 
\hypertarget{interface_a_t_resource_client_a53a5d620255bcc5dc198b4106f8141d3}{
(void) -\/ {\bfseries didFetchItem:withURI:}}
\label{interface_a_t_resource_client_a53a5d620255bcc5dc198b4106f8141d3}

\item 
\hypertarget{interface_a_t_resource_client_ac12de9fa49fa0340ac7f1d168214c986}{
(void) -\/ {\bfseries saveObject:}}
\label{interface_a_t_resource_client_ac12de9fa49fa0340ac7f1d168214c986}

\item 
\hypertarget{interface_a_t_resource_client_ac7150d971b28b7287523db7f29b76972}{
(void) -\/ {\bfseries saveObject:options:}}
\label{interface_a_t_resource_client_ac7150d971b28b7287523db7f29b76972}

\item 
\hypertarget{interface_a_t_resource_client_ae0201c8558a2e74caa524f0790c2bcef}{
(void) -\/ {\bfseries loadRoute:params:delegate:}}
\label{interface_a_t_resource_client_ae0201c8558a2e74caa524f0790c2bcef}

\item 
\hypertarget{interface_a_t_resource_client_aa9c928616e58786bdb15973e24fcd1af}{
(void) -\/ {\bfseries loadRoutesFromResource:}}
\label{interface_a_t_resource_client_aa9c928616e58786bdb15973e24fcd1af}

\item 
\hypertarget{interface_a_t_resource_client_a53ce9d8ae5a79f9a2f69cc861744109a}{
(\hyperlink{struct___a_t_route}{ATRoute}) -\/ {\bfseries routeForEntity:action:}}
\label{interface_a_t_resource_client_a53ce9d8ae5a79f9a2f69cc861744109a}

\item 
\hypertarget{interface_a_t_resource_client_aca4c252b5fc7af7ea5a41044665b6332}{
(\hyperlink{struct___a_t_route}{ATRoute}) -\/ {\bfseries routeForEntity:action:params:}}
\label{interface_a_t_resource_client_aca4c252b5fc7af7ea5a41044665b6332}

\end{DoxyCompactItemize}
\subsubsection*{Properties}
\begin{DoxyCompactItemize}
\item 
\hypertarget{interface_a_t_resource_client_a799228e0a82deb990c3fc58d69a212b3}{
\hyperlink{interface_a_t_synchronizer}{ATSynchronizer} $\ast$ {\bfseries sync}}
\label{interface_a_t_resource_client_a799228e0a82deb990c3fc58d69a212b3}

\item 
\hypertarget{interface_a_t_resource_client_afa980946f714b777325182bfa38da016}{
RKClient $\ast$ {\bfseries client}}
\label{interface_a_t_resource_client_afa980946f714b777325182bfa38da016}

\item 
\hypertarget{interface_a_t_resource_client_a12baa61aa58e54ccb95025ade33712d3}{
NSDictionary $\ast$ {\bfseries routes}}
\label{interface_a_t_resource_client_a12baa61aa58e54ccb95025ade33712d3}

\item 
\hypertarget{interface_a_t_resource_client_a47f4ef4c9464b546ba7b39eb212f2919}{
\hyperlink{class_n_s_string}{NSString} $\ast$ {\bfseries IDField}}
\label{interface_a_t_resource_client_a47f4ef4c9464b546ba7b39eb212f2919}

\end{DoxyCompactItemize}


The documentation for this class was generated from the following files:\begin{DoxyCompactItemize}
\item 
ATResourceClient.h\item 
ATResourceClient.m\end{DoxyCompactItemize}

\hypertarget{interface_a_t_synchronizer}{
\subsubsection{ATSynchronizer Class Reference}
\label{interface_a_t_synchronizer}\index{ATSynchronizer@{ATSynchronizer}}
}
Inheritance diagram for ATSynchronizer:\begin{figure}[h]
\begin{center}
\leavevmode
\includegraphics[height=2.000000cm]{interface_a_t_synchronizer}
\end{center}
\end{figure}
\subsubsection*{Public Member Functions}
\begin{DoxyCompactItemize}
\item 
\hypertarget{interface_a_t_synchronizer_a1a230c698ce38b033b22310db13120f5}{
(id) -\/ {\bfseries initWithAppContext:}}
\label{interface_a_t_synchronizer_a1a230c698ce38b033b22310db13120f5}

\item 
\hypertarget{interface_a_t_synchronizer_aa091026c249106b0f059a86ae9c5afd8}{
(void) -\/ {\bfseries close}}
\label{interface_a_t_synchronizer_aa091026c249106b0f059a86ae9c5afd8}

\item 
\hypertarget{interface_a_t_synchronizer_affb70ca203d9a187417db200776685c0}{
(\hyperlink{class_n_s_string}{NSString} $\ast$) -\/ {\bfseries authKeyOrNull}}
\label{interface_a_t_synchronizer_affb70ca203d9a187417db200776685c0}

\item 
\hypertarget{interface_a_t_synchronizer_a6b5f15941c11220049e877543e7adb62}{
(void) -\/ {\bfseries fetchEntity:}}
\label{interface_a_t_synchronizer_a6b5f15941c11220049e877543e7adb62}

\item 
\hypertarget{interface_a_t_synchronizer_ae7b25a346a2082dfbd5e54ac9ae6619f}{
(void) -\/ {\bfseries syncObject:}}
\label{interface_a_t_synchronizer_ae7b25a346a2082dfbd5e54ac9ae6619f}

\item 
\hypertarget{interface_a_t_synchronizer_adf5b437093a61f09ee914cc61e30f62d}{
(void) -\/ {\bfseries startSync}}
\label{interface_a_t_synchronizer_adf5b437093a61f09ee914cc61e30f62d}

\item 
\hypertarget{interface_a_t_synchronizer_add32dc8f725e3d33a8b49e9bea0460ba}{
(void) -\/ {\bfseries sync}}
\label{interface_a_t_synchronizer_add32dc8f725e3d33a8b49e9bea0460ba}

\item 
\hypertarget{interface_a_t_synchronizer_a063d9fcbbadfd834fded74a9a90de04e}{
(void) -\/ {\bfseries updateObjectAtURI:withDictionary:}}
\label{interface_a_t_synchronizer_a063d9fcbbadfd834fded74a9a90de04e}

\item 
\hypertarget{interface_a_t_synchronizer_a5975c374e80ef16d1131b8af6a3f1c17}{
(void) -\/ {\bfseries changeURIFrom:to:}}
\label{interface_a_t_synchronizer_a5975c374e80ef16d1131b8af6a3f1c17}

\item 
\hypertarget{interface_a_t_synchronizer_a237c7017d5b328901ef056c4174b1284}{
(void) -\/ {\bfseries startAutosync}}
\label{interface_a_t_synchronizer_a237c7017d5b328901ef056c4174b1284}

\item 
\hypertarget{interface_a_t_synchronizer_a50d1711cbe2f564ea1633e80fbc00dd7}{
(void) -\/ {\bfseries stopAutosync}}
\label{interface_a_t_synchronizer_a50d1711cbe2f564ea1633e80fbc00dd7}

\item 
\hypertarget{interface_a_t_synchronizer_a311d78f60c5b19b749bcc47ec43aa81b}{
(void) -\/ {\bfseries \_\-didChangeAppObject:}}
\label{interface_a_t_synchronizer_a311d78f60c5b19b749bcc47ec43aa81b}

\end{DoxyCompactItemize}
\subsubsection*{Protected Attributes}
\begin{DoxyCompactItemize}
\item 
\hyperlink{interface_a_t_mapping_helper}{ATMappingHelper} $\ast$ \hyperlink{interface_a_t_synchronizer_a31d09b73a13ca9815cd0f62d60dbc373}{\_\-mappingHelper}
\item 
\hyperlink{interface_a_t_meta_context}{ATMetaContext} $\ast$ \hyperlink{interface_a_t_synchronizer_a14e16b95fa385616596db1b540fb8280}{\_\-metaContext}
\item 
\hypertarget{interface_a_t_synchronizer_a4d68957b743ed4634a79496c9ce4857e}{
\hyperlink{interface_a_t_app_context}{ATAppContext} $\ast$ {\bfseries \_\-appContext}}
\label{interface_a_t_synchronizer_a4d68957b743ed4634a79496c9ce4857e}

\item 
\hyperlink{interface_a_t_message_client}{ATMessageClient} $\ast$ \hyperlink{interface_a_t_synchronizer_a7540768d28730fa0da9a29c8bd3db701}{\_\-messageClient}
\item 
\hypertarget{interface_a_t_synchronizer_ac8d7d4aca2cae90b6b218443d9dbcb77}{
\hyperlink{interface_a_t_resource_client}{ATResourceClient} $\ast$ {\bfseries \_\-resourceClient}}
\label{interface_a_t_synchronizer_ac8d7d4aca2cae90b6b218443d9dbcb77}

\item 
\hyperlink{class_n_s_string}{NSString} $\ast$ \hyperlink{interface_a_t_synchronizer_aff441de7abd4bb9fb6fb62ebfb36c072}{\_\-authKey}
\item 
\hypertarget{interface_a_t_synchronizer_a2f1074693dc133a3a76305842905cbe7}{
BOOL {\bfseries \_\-isSyncScheduled}}
\label{interface_a_t_synchronizer_a2f1074693dc133a3a76305842905cbe7}

\end{DoxyCompactItemize}
\subsubsection*{Properties}
\begin{DoxyCompactItemize}
\item 
\hypertarget{interface_a_t_synchronizer_aa4d38b9c8291f814c3e89a936e366748}{
\hyperlink{interface_a_t_meta_context}{ATMetaContext} $\ast$ {\bfseries metaContext}}
\label{interface_a_t_synchronizer_aa4d38b9c8291f814c3e89a936e366748}

\item 
\hypertarget{interface_a_t_synchronizer_ae4a168720bbd872e1fb509a951edc6f0}{
\hyperlink{interface_a_t_app_context}{ATAppContext} $\ast$ {\bfseries appContext}}
\label{interface_a_t_synchronizer_ae4a168720bbd872e1fb509a951edc6f0}

\item 
\hypertarget{interface_a_t_synchronizer_ad8ac915848318015ef77a42d367a4a64}{
\hyperlink{interface_a_t_mapping_helper}{ATMappingHelper} $\ast$ {\bfseries mappingHelper}}
\label{interface_a_t_synchronizer_ad8ac915848318015ef77a42d367a4a64}

\item 
\hypertarget{interface_a_t_synchronizer_a5506f23cbd14b9a27c6749a7eaf95d20}{
\hyperlink{interface_a_t_message_client}{ATMessageClient} $\ast$ {\bfseries messageClient}}
\label{interface_a_t_synchronizer_a5506f23cbd14b9a27c6749a7eaf95d20}

\item 
\hypertarget{interface_a_t_synchronizer_a11fbd96e8afa8f5a36c1be409d447053}{
\hyperlink{interface_a_t_resource_client}{ATResourceClient} $\ast$ {\bfseries resourceClient}}
\label{interface_a_t_synchronizer_a11fbd96e8afa8f5a36c1be409d447053}

\item 
\hypertarget{interface_a_t_synchronizer_aa10ad4a0dfe439088079385499f8e0ec}{
\hyperlink{class_n_s_string}{NSString} $\ast$ {\bfseries authKey}}
\label{interface_a_t_synchronizer_aa10ad4a0dfe439088079385499f8e0ec}

\item 
id$<$ \hyperlink{protocol_a_t_synchronizer_delegate-p}{ATSynchronizerDelegate} $>$ \hyperlink{interface_a_t_synchronizer_abef36e70ffe47396911ae10ed7c565a7}{delegate}
\end{DoxyCompactItemize}


\subsubsection{Member Data Documentation}
\hypertarget{interface_a_t_synchronizer_aff441de7abd4bb9fb6fb62ebfb36c072}{
\index{ATSynchronizer@{ATSynchronizer}!\_\-authKey@{\_\-authKey}}
\index{\_\-authKey@{\_\-authKey}!ATSynchronizer@{ATSynchronizer}}
\subsubsection[{\_\-authKey}]{\setlength{\rightskip}{0pt plus 5cm}-\/ ({\bf NSString}$\ast$) {\bf \_\-authKey}\hspace{0.3cm}{\ttfamily  \mbox{[}protected\mbox{]}}}}
\label{interface_a_t_synchronizer_aff441de7abd4bb9fb6fb62ebfb36c072}
State \hypertarget{interface_a_t_synchronizer_a31d09b73a13ca9815cd0f62d60dbc373}{
\index{ATSynchronizer@{ATSynchronizer}!\_\-mappingHelper@{\_\-mappingHelper}}
\index{\_\-mappingHelper@{\_\-mappingHelper}!ATSynchronizer@{ATSynchronizer}}
\subsubsection[{\_\-mappingHelper}]{\setlength{\rightskip}{0pt plus 5cm}-\/ ({\bf ATMappingHelper}$\ast$) {\bf \_\-mappingHelper}\hspace{0.3cm}{\ttfamily  \mbox{[}protected\mbox{]}}}}
\label{interface_a_t_synchronizer_a31d09b73a13ca9815cd0f62d60dbc373}
Helper \hypertarget{interface_a_t_synchronizer_a7540768d28730fa0da9a29c8bd3db701}{
\index{ATSynchronizer@{ATSynchronizer}!\_\-messageClient@{\_\-messageClient}}
\index{\_\-messageClient@{\_\-messageClient}!ATSynchronizer@{ATSynchronizer}}
\subsubsection[{\_\-messageClient}]{\setlength{\rightskip}{0pt plus 5cm}-\/ ({\bf ATMessageClient}$\ast$) {\bf \_\-messageClient}\hspace{0.3cm}{\ttfamily  \mbox{[}protected\mbox{]}}}}
\label{interface_a_t_synchronizer_a7540768d28730fa0da9a29c8bd3db701}
Networking clients \hypertarget{interface_a_t_synchronizer_a14e16b95fa385616596db1b540fb8280}{
\index{ATSynchronizer@{ATSynchronizer}!\_\-metaContext@{\_\-metaContext}}
\index{\_\-metaContext@{\_\-metaContext}!ATSynchronizer@{ATSynchronizer}}
\subsubsection[{\_\-metaContext}]{\setlength{\rightskip}{0pt plus 5cm}-\/ ({\bf ATMetaContext}$\ast$) {\bf \_\-metaContext}\hspace{0.3cm}{\ttfamily  \mbox{[}protected\mbox{]}}}}
\label{interface_a_t_synchronizer_a14e16b95fa385616596db1b540fb8280}
Context 

\subsubsection{Property Documentation}
\hypertarget{interface_a_t_synchronizer_abef36e70ffe47396911ae10ed7c565a7}{
\index{ATSynchronizer@{ATSynchronizer}!delegate@{delegate}}
\index{delegate@{delegate}!ATSynchronizer@{ATSynchronizer}}
\subsubsection[{delegate}]{\setlength{\rightskip}{0pt plus 5cm}-\/ (id$<$ {\bf ATSynchronizerDelegate} $>$) delegate\hspace{0.3cm}{\ttfamily  \mbox{[}read, write, assign\mbox{]}}}}
\label{interface_a_t_synchronizer_abef36e70ffe47396911ae10ed7c565a7}
Delegate 

The documentation for this class was generated from the following files:\begin{DoxyCompactItemize}
\item 
ATSynchronizer.h\item 
ATSynchronizer.m\end{DoxyCompactItemize}

\hypertarget{protocol_a_t_synchronizer_delegate-p}{
\subsubsection{$<$ATSynchronizerDelegate$>$ Protocol Reference}
\label{protocol_a_t_synchronizer_delegate-p}\index{ATSynchronizerDelegate-\/p@{ATSynchronizerDelegate-\/p}}
}
\subsubsection*{Public Member Functions}
\begin{DoxyCompactItemize}
\item 
\hypertarget{protocol_a_t_synchronizer_delegate-p_a5ef10b52a53f28e5f16cb76c2e9af9b8}{
(void) -\/ {\bfseries clientAuthDidSucceed:}}
\label{protocol_a_t_synchronizer_delegate-p_a5ef10b52a53f28e5f16cb76c2e9af9b8}

\item 
\hypertarget{protocol_a_t_synchronizer_delegate-p_a25d978a721f2bd286b9e063a1987f45d}{
(void) -\/ {\bfseries clientAuthDidFail:}}
\label{protocol_a_t_synchronizer_delegate-p_a25d978a721f2bd286b9e063a1987f45d}

\end{DoxyCompactItemize}


The documentation for this protocol was generated from the following file:\begin{DoxyCompactItemize}
\item 
ATSynchronizer.h\end{DoxyCompactItemize}

%
%
% \ititnclude{appendixc}
%% begin the 'Curriculumvitae' of the author
% \curriculumvitae\protect\label{page:posledna}
% Táto časť\/ je nepovinná. Autor tu môže uviesť\/ svoje biografické
% údaje, údaje o~záujmoch, účasti na~projektoch, účasti na~súťažiach,
% získané ocenenia, zahraničné pobyty na~praxi, domácu prax, publikácie
% a~pod.
% \endcurriculumvitae

\end{document}
%%