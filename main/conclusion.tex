\section{Conclusion}

The thesis starts by analyzing current state of cloud and its client applications. The result is that there is still a lot of improvement space. Atmosphere focuses on reducing the waiting time as perceived by the user. Additionally, it focuses on providing real-time collaborative environment for applications.

Once the problem is defined, the thesis continues by analyzing available technologies that could be used to solve it. There are many new HTML5 features such as WebSockets and local storage that could be used for building Atmosphere. 

The next part introduces the basic ideas behind design of Atmosphere. It's concluded that the design should allow working with an external REST API, it should cache objects locally and synchronize them in background by keeping track of changed objects.

According to the design a implementation is described, such as specific data structures and algorithms. Atmosphere is implemented for two main platforms: The web (JavaScript) and Cocoa (Mac, iPhone and iPad applications). The rest of the chapter describes implementation specifics for individual platforms.

Next follows a brief description of how was Atmosphere used in three client applications for cloud services. The chapter contains real-world backing for some of the decisions made in the design phase, as well as description of new problems encountered while implementing these applications.

The work is concluded by comparing Atmosphere to other products. All of these products have existed for only a couple of months which shows how current a problem it is that Atmosphere solves.