\section{Conclusion}

The aim of this work was to describe all aspects of building Atmosphere, a data synchronization platform for web and Mac/iOS applications.

The first chapters started by general analysis of the client applications, their strong and weak points. The result was that there is still a lot of space to improve, specifically in objects caching. Most of the applications require users to wait for network requests to finish, even when it is not needed. It was also concluded that only some of the applications provide concurrent access by exchanging data in real-time. This features should be supported by all applications and Atmosphere allows that.

Once the problem was defined, the thesis continued by analyzing related technologies. There are many new HTML5 features such as WebSockets, or local storage that are utilized by Atmosphere. Since Atmosphere is not only web applications platform, technologies in the area of desktop and mobile applications were analyzed too, specifically the Mac and iOS platforms.

The next part introduced basic ideas behind the design of Atmosphere. Basic design decisions were concluded, such as that Atmosphere should allow working with an existing server application, it should cache objects locally and synchronize them in background by keeping track of changed objects. Also, a high-level layout of components was outlined.

The platform was implemented for two environments, the web (JavaScript) and Mac/iOS (Cocoa). The thesis contains details of these implementations, such as specific data structures and algorithms. Also, for each platform, its specifics were described too.

The part of the implementation work was to build example applications. Three applications with different technological background were built. The thesis contains real-world backing for some of the decisions made in the design phase, as well as description of new problems encountered while implementing these applications.

The work was concluded by comparing Atmosphere to other similar products. None of these products existed before the work on Atmosphere started, which shows how fast is the development in the area of SaaS applications.

Atmosphere is not the only project that uses local caching to improve user experience, but differs from others by the ability to use it with existing server and being open-source.