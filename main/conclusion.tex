\section{Conclusion}

This thesis is focused on client applications for SaaS products. It starts by general analysis of these applications. The result is that there is still a lot of improvement space. Most of the applications require users to wait for network requests to finish, even when it's not needed. Also, only some of the application provide concurrent access by real-time data synchronization.

Once the problem is defined, the thesis continues by analyzing available technologies that could be used to solve it. There are many new HTML5 features such as WebSockets and local storage that could be used for building Atmosphere. Building desktop applications is a viable solution too.

The next part introduces the basic ideas behind the design of Atmosphere. It is concluded that the design should allow working with an external REST API, it should cache objects locally and synchronize them in background by keeping track of changed objects. A high-level structure of the libraries is outlined too.

The framework is implemented for two environments, the web (JavaScript) and Mac/iOS (Cocoa). The details of this work, such as specific data structures and algorithms are described in the next chapter. The rest of the chapter describes implementation specifics for individual platforms.

Next follows a brief description of how was Atmosphere used in three client applications for cloud services. The chapter contains real-world backing for some of the decisions made in the design phase, as well as description of new problems encountered while implementing these applications.

The work is concluded by comparing Atmosphere to other products. All of these products have existed for only a couple of months which shows that this area is in a very fast development.