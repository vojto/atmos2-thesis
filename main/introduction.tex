\setcounter{page}{1}
\setcounter{equation}{0}
\setcounter{figure}{0}
\setcounter{table}{0}

\section*{Introduction}
\addcontentsline{toc}{section}{\numberline{}Introduction}

The basic idea behind Atmosphere is to make client applications of SaaS (Software as a service) products more user friendly.

The first chapter defines cloud computing and individual types of cloud solutions.

The next chapter describes the motivation behind Atmosphere. It talks about the current state of client applications and how they could be improved. Then available technologies are described, specifically HTML5 and Cocoa.

This thesis describes Atmosphere, a platform that improves user experience of web, mobile and desktop applications by storing remote objects in local database.

The next part proposes high level design of Atmosphere. It discusses the basic idea of asynchronous interfaces, how local storage could be used, how to recover from failure and also what other design solutions were considered and why the one described was chosen.

The next chapter is about the building blocks of Atmosphere: classes, data structures and algorithms. It is a platform-independent description, while platform specifics are described by the end of the chapter.

Use cases describe specific application that use Atmosphere: TaskDo, Edukit and Zone. They cover a large portion of Atmosphere use cases including JavaScript and Cocoa applications, using custom or third-party REST APIs.

The thesis is concluded with description of other commercial or open-source solutions that were recently released to public. These products are all compared with each other and with Atmosphere. 