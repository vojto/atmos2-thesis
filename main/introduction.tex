\setcounter{page}{1}
\setcounter{equation}{0}
\setcounter{figure}{0}
\setcounter{table}{0}

\section*{Introduction}
\addcontentsline{toc}{section}{\numberline{}Introduction}

The basic idea behind Atmosphere is to make SaaS (Software as a service) client applications more user friendly by caching objects locally, synchronizing them in the background and propagate changes between users in real-time.

At the time of writing there were already projects solving this problem, but only partially. Atmosphere aims to create an integrated solution, that would work on both web and Mac/iOS platforms, allow offline usage, and be open-source. Additionally, it should work with an existing REST server, so developers wouldn't have to rewrite their code from scratch. The results should be released as open-source and provide an easy way to build new applications.

The area of SaaS client applications is under fast development, specifically the area of JavaScript applications. Some new projects similar to Atmosphere with strong focus on JavaScript implementation were released during the time of writing. Atmosphere still provides some unique features (namely support for custom server) as described in the last chapter.

The overall goal for Atmosphere is to provide a platform that would allow building very user friendly client applications with minimal effort. It is best-suited for cases when the sever code is already available.

The first chapter defines cloud computing and individual types of cloud solutions. The next chapter talks about the current state of client applications and how they could be improved. Then available technologies are described, specifically HTML5 and Cocoa.

The next part proposes high level design of Atmosphere. It discusses the basic idea of asynchronous interfaces, how local storage could be used, how to recover from failure and also what other design solutions were considered and why the one described was chosen.

The next chapter is about the building blocks of Atmosphere: classes, data structures and algorithms. It is a platform-independent description, while platform specifics are described by the end of the chapter.

Use cases describe specific application that use Atmosphere: TaskDo, Edukit and Zone. They cover a large portion of Atmosphere use cases including JavaScript and Cocoa applications, using custom or third-party REST APIs.

The thesis is concluded with description of other open or closed-source solutions that were recently released to the public. These products are all compared with each other and with Atmosphere. 