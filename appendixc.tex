\section*{Appendix C}
\addcontentsline{toc}{section}{\numberline{}Appendix C}
%\appendix
%\section{Príloha}
\subsection*{Vytvorenie zoznamu skratiek a symbolov}

Ak sú v~práci skratky a symboly, vytvára sa {\it Zoznam skratiek
a~symbolov} ({\it List of Symbols and Abbreviations}) a~ich
dešifrovanie. V~prostredí \LaTeX{}u sa takýto zoznam ľahko vytvorí
pomocou balíka \verb+nomencl+. Postup je nasledovný:
\begin{enumerate}
\item Do preambuly zapíšeme nasledujúce príkazy\\
\verb+\usepackage[noprefix]{nomencl}+\\ \verb+\makeglossary+
\item  V~mieste, kde má byť\/ vložený zoznam zapíšeme príkaz\\
\verb+\printglossary+
\item V miestach, kde sa vyskytujú skratky a symboly ich definíciu
zavedieme, napr. ako v~našom texte, príkazmi\\
\verb+\nomenclature{$\upmu$}{mikro, $10^{-6}$}+\\
\verb+\nomenclature{V}{volt, základná jednotka napätia v sústave SI}+\\
a~dokument \uv{pre\LaTeX{}ujeme}.
\item Z~príkazového riadka spustíme program \verb+makeindex+
s~prepínačmi podľa použitého operačného systému, napr.~pre OS GNU/Linux
a~distribúciu~Ubuntu~$6.06$~LTS s~verziou \verb+tetex 3.0-15ubuntu1+
napíšeme:\\
\verb*+makeindex tukethesis.glo -s nomencl.ist -o tukethesis.gls+
\item Po opätovnom \uv{pre\LaTeX{}ovaní} dokumentu sa na
požadované miesto vloží {\it Zoznam skratiek a symbolov} ({\it List
of Symbols and Abbreviations}).
\end{enumerate}
