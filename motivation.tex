\section{Motivation}

During the last couple of years a lot of software has migrated from desktop application to web applications.

Many tasks can be now performed using only the web browser. A project called Chromebook \ref{chromebook} introduced a laptop which contains only a web browser and all the work is done using web applications.

That makes development and use of software better in many ways: Web applications are faster to implement, it’s easy to customize interfaces, there is no such thing as software updates anymore, and the data is stored and processed by the server.

Data being stored on the server are safer, but also it allows sharing of data and communication.

\subsection{Client applications}

\begin{figure}[ht!]
\centering
\includegraphics{Motivation_1.png}
\caption{Diagram of clients in cloud architecture \label{fig:1}}
\end{figure}

Client is the computer hardware that is responsible for accessing the cloud. Application is what is provided by the cloud service. The definition of SaaS implies, that user should be able to use the application by signing up for the service, which is in most cases possible, because applications are delivered through the web.

However, in some cases further installation is required. For example, on iPhone platform, web applications are limited, they can’t access some hardware features such as camera. If the cloud application would require taking pictures, application would have to be installed through the App Store, which is installation of software.
 	
For SaaS, if native client application is required, it’s usually free: The business model around SaaS is to charge user for using the service, not for installing software. This is still SaaS, because user can freely download the client application and any upgrades, but it will render useless unless they subscribe to the service.  
	
The goal of Atmosphere is to improve these client applications, but first it's needed to take a look at the current state and find out what exactly needs improving.

\subsubsection{Desktop applications}

The biggest issue with desktop applications is that they are missing. Classic desktop applications have their place: It’s the once that do a little networking and can be hardly called SaaS. But SaaS seem to choose to build web application instead of desktop applications.

They actually have a good reason to do so: Desktop applications require a lot more work, especially when they should run on all platforms. In most cases it’s probably not worth investing money into building individual applications for every platforms, but it some cases, especially in products that are used frequently, having a desktop applications means much better experience, much less time lost waiting for network request to complete and much higher quality of the software.

One of the great examples are the social media applications. With millions of users and open API  it is only logical that developers try to improve the experience by building a desktop applications. As for a particular example, Twitter has tens of desktop clients for all Windows, Mac and Linux platforms.

It seems that user experience is generally better with desktop application rather than web applications. The first reason is that the user interface is on the client side: That means there are no requests, there is no need for AJAX, only the data is transferred through the network. Some desktop applications even use local storage to cache the local data.

So the only problem left to solve is the very case when application doesn’t work with local data, but always makes networks requests, blocking the interface, needlesly making user wait. 

The desktop library of Atmosphere solves this problem by providing a library that extends local storage framework and adds transparent networking without interface blocking. 

One last type of desktop applications are those with purpose of notifying users of events. Should an event occur in a web application, the only way of immediately notifying user is via email. These emails tend to get very annoyoing after time, so user start creating filters that hide them from the inbox, which partly defies the purpose.

A better way of notifying users is to provide a desktop application they can launch and quit. Part of Atmosphere is the notification server which together with client libraries allow instant notifications for events that occured in the application. 

\subsubsection{Mobile applications}

Mobile applications became a very important part of any SaaS. It is time when many users own smartphones with constant connection to the Internet, and they expect every service to provide a mobile application, that would allow them to perform the basic  tasks using a simplified user interface. 

Mobile applications are usually a top priority when a new SaaS product is shipped. Regular, web-based product is unusable on a small mobile phone. A special, mobile version of the site could solve the problem, but compared to native mobile apps, these lack a great deal of responsiveness and speed.

Also, a native mobile application provides access to hardware-specific features, such as the the camera, GPS, the accelometers, multi-touch event handling, and others. 

So what problems do the mobile applications have. Ironically, it seems that these are most evolved. Compared to desktop applications, they are built a lot more frequently. Compared to web applications, they are much more responsive and faster. 

There is even an excellent framework RestKit [10] that allows building iOS applications based on REST web services. It also provides a way of caching retrieved data in a local database.

The only flaw of this framework is that it doesn’t support creating objects locally, when there’s no internet connection, and then synchronizing with the web service. Atmosphere solves this problem by tracking local changes all the time, no matter if user has internet connection or not, and synchronizing them with the web service when there is connection available.

Notifications are supported on mobile platforms in different ways. On iOS there’s a central notification system, that is managed by Apple infrastructure. The application is not responsible for managing open connection to receive notifications, instead when an event occures on the server side, it notifies the Apple notification service that then delivers notification to the client. This solution is superior to managing connection by every client connection individually. First, it saves the memory footprint, CPU usage, thus it saves battery life. Secondly, the code responsible for managing the connection is provided and well tested by Apple. Network programming is a very unstable, so it would take a lot of effort to come up with solution as reliable as Apple’s. 

Atmosphere won’t deal with push notifications in a way of receiving notifications when application is not active. But it will apply the same solution of updating objects as soon as they are changed on a different machine to mobile applications. 

\subsubsection{Web applications}

There is one significant difference between desktop/mobile apps and those that run in a browser. Native apps ship a package of client code that needs to be installed and run on the client platform. Web applications, in contrast, can run anywhere without any installation.

About the greatest issue with current web applications is that they are not really applications. Most of them is just a set of web pages and forms. Some use AJAX to make requests in the background, but there is still occasional page reload.

That requires most of the logic on the server side while only the results of requests are delivered to user. This concept is old and must be replaced by JavaScript-heavy client side applications that have their own logic, especially around views. They would make request only when necessary.

This thesis doesn’t explore all of the possible solutions of achieving this goal for there are many various options. Instead, Atmosphere assumes the developer has or is building a JavaScript application.

Moving logic to the client side is a great improvement itself. Still there are the same problems as mentioned before. For example, TaskDo (one of Atmosphere-based applications, that were build along with the framework), provides an easy way to access user’s tasks in Google Tasks. The simple interface consists of two screens: List of „task lists“ and list of tasks in the list.

Without Atmosphere, clicking on a task list would result in a few seconds wait for the tasks inside to load. With Atmosphere, the data is already cached locally, so opening a task list is instant. TaskDo and other applications are described in section Case Study.

The next problem with web applications is that there are no push notifications, unless the business domain explicitly requires them. (Such as a chat application, or a game.) For most applications, it’s not exactly necessary to update objects  in real time between users, but it would make the application better.

\subsection{Conclusion}

It’s the current problems with client applications what Atmosphere is trying to remove. This section discussed what are the problems that make current client applications inconvenient to use and slightly outlined the proposed solutions. In further sections we’ll discuss the technologies that will be used and the solutions themselves.